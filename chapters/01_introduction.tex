% !TeX root = ../main.tex
% Add the above to each chapter to make compiling the PDF easier in some editors.

\chapter{Introduction}\label{chapter:introduction}

With the rapid development of communication technologies, social media has become one of the most frequently used news sources. For example, a study from Pew Research Center\footnote{https://www.pewresearch.org/journalism/2021/09/20/news-consumption-across-social-media-in-2021/} reports that in 2021, 48\% of U.S. adults get their news from social media “often” or “sometimes”. As another example, global data from 2022\footnote{https://www.statista.com/statistics/718019/social-media-news-source/} shows that over 70\% of adults from Kenya, Malaysia, Phillippines, Bulgaria, and Greece use social media as one of their news sources, while this share is lower than 40\% for the adults in the United Kingdom, The Netherlands, Germany, and Japan. These statistics indicate that social media is a crucial news source. Still, one should follow the news on social media carefully since there is no regulatory authority to check the news. For instance, a study in Digital News Report 2022~\parencite{ReutersInstituteDigitalNewsReport}  reports in its key findings that trust in the news is 42\% globally, the highest (69\%) in Finland, and the lowest (26\%) in the U.S.A. Consequently, the same study shows in early 2022, in the week of the survey, between 45\% and 55\% of surveyed social media consumers worldwide have witnessed false or misleading information about COVID-19.

The term fake news has existed for over a century~\parencite{TheScienceOfFakeNews_Lazer}.  It rose to prominence with the 2016 U.S. Presidential Election~\parencite{USPresidentialElection2016}. With its prevalence, the research community has focused on the automatic detection of fake news~\parencite{FakeNewsDetectionOnSocialMedia_Shu, FakeNewsNet_Shu,  LiarLiarPantsOnFire, FakeNewsDetectionUsingGeometricDeepLearning, SupervisedLearningForFakeNewsDetection}.



\subsection{Subsection}

See~\autoref{tab:sample}, \autoref{fig:sample-drawing}, \autoref{fig:sample-plot}, \autoref{fig:sample-listing}.

\begin{table}[htpb]
  \caption[Example table]{An example for a simple table.}\label{tab:sample}
  \centering
  \begin{tabular}{l l l l}
    \toprule
    A & B & C & D \\
    \midrule
    1 & 2 & 1 & 2 \\
    2 & 3 & 2 & 3 \\
    \bottomrule
  \end{tabular}
\end{table}

\begin{figure}[htpb]
  \centering
  % This should probably go into a file in figures/
  \begin{tikzpicture}[node distance=3cm]
    \node (R0) {$R_1$};
    \node (R1) [right of=R0] {$R_2$};
    \node (R2) [below of=R1] {$R_4$};
    \node (R3) [below of=R0] {$R_3$};
    \node (R4) [right of=R1] {$R_5$};

    \path[every node]
    (R0) edge (R1)
    (R0) edge (R3)
    (R3) edge (R2)
    (R2) edge (R1)
    (R1) edge (R4);
  \end{tikzpicture}
  \caption[Example drawing]{An example for a simple drawing.}\label{fig:sample-drawing}
\end{figure}

\begin{figure}[htpb]
  \centering

  \pgfplotstableset{col sep=&, row sep=\\}
  % This should probably go into a file in data/
  \pgfplotstableread{
    a & b    \\
    1 & 1000 \\
    2 & 1500 \\
    3 & 1600 \\
  }\exampleA
  \pgfplotstableread{
    a & b    \\
    1 & 1200 \\
    2 & 800 \\
    3 & 1400 \\
  }\exampleB
  % This should probably go into a file in figures/
  \begin{tikzpicture}
    \begin{axis}[
        ymin=0,
        legend style={legend pos=south east},
        grid,
        thick,
        ylabel=Y,
        xlabel=X
      ]
      \addplot table[x=a, y=b]{\exampleA};
      \addlegendentry{Example A};
      \addplot table[x=a, y=b]{\exampleB};
      \addlegendentry{Example B};
    \end{axis}
  \end{tikzpicture}
  \caption[Example plot]{An example for a simple plot.}\label{fig:sample-plot}
\end{figure}

\begin{figure}[htpb]
  \centering
  \begin{tabular}{c}
    \begin{lstlisting}[language=SQL]
    SELECT * FROM tbl WHERE tbl.str = "str"
  \end{lstlisting}
  \end{tabular}
  \caption[Example listing]{An example for a source code listing.}\label{fig:sample-listing}
\end{figure}
