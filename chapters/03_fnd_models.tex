% !TeX root = ../main.tex
% Add the above to each chapter to make compiling the PDF easier in some editors.

\chapter{Fake News Detection Models}\label{chapter:fnd_models}

\section{Content Based Models}

\subsection{Definitons}
Talk about text based models, tf-idf, bag of words(BoW), how BERT is used in these tasks, (in the end) just assert that only text based models are not sufficient.

\subsection{Dataset}
Used the Kaggle competition dataset.
-> Talk about the general analysis of the dataset. (How many instances, real/fake instances, )

\subsection{Tokenizer}
Used DistilRoBERTa tokenizer. (check the tokenizer of the model and talk about it)

\subsection{Model}
Used the model in transformers repository. The model from GonzaloA was used since it also provided its dataset and their train/val/test splits.

\subsection{Explainability and Explanation}
The model seems to have memorized some basic patterns and rely on that.
Talk about the properties of explanation techniques. (Localization, )
Define explainability. Define explanation.
Input perturbation
Explain a novel news (use test data)

\section{Social Context Based Models}
Talk about models that incorporate social context, spatiotemporal information and other context with text data. Can be any kind of model.

\subsection{Geometric Deep Learning}
Talk about Graph Neural Networks

\subsection{Dataset}
FakeNewsNet, UPFD, explain the dataset, no of edges/nodes. Which models use this dataset,

\subsection{Models}
SAGE GNN
UPFD GCNFN
